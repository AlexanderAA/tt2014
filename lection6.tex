\section{О классе функций, определимых в просто типизированном лямбда-исчислении}

\begin{definition}
Назовем расширенным полиномом функцию 
\end{definition}

\begin{lemma}
Если в выражении $X^\xi$, находящемся в нормальной форме,
подтерм $T^\tau$ не является свободной переменной выражения
$T$, и $T \ne X$, то всегда найдется такой подтерм $S^\sigma$, что 
$h(\sigma) > h(\tau)$, причем $\sigma=\tau\rightarrow\rho$ или
$\sigma=\rho\rightarrow\tau$.
\end{lemma}

\begin{proof}
Рассмотрим подтерм $T$. Возможны следующие варианты:

\begin{enumerate}
\item $T$ --- это некоторая переменная $x$ (она обязана быть связанной по условию леммы).
То есть $T$ --- часть выражения $S = \lambda x:\tau. \dots x \dots$. 
Тогда $S: \tau\rightarrow\rho$, и $h(\tau\rightarrow\rho) > h(\tau)$.
\item $T$ --- это некоторая абстракция $T=\lambda x:\sigma.P^\pi$. 
Тогда заметим, что по условию $T \ne X$. Значит, $T$ входит в некоторое более
общее выражение --- либо в абстракцию
$S^{\upsilon\rightarrow\tau} = \lambda y:\upsilon.T$, либо
в применение $S^{\tau\rightarrow\upsilon} T$
(применение вида $T A$ является редексом и потому невозможно).
\item $T$ --- это некоторое применение $T=S^\upsilon\rightarrow\tau Y$.
\end{enumerate}

\end{proof}

\begin{theorem}
При фиксированном типе для целых чисел 
$\nu=(\alpha\rightarrow\alpha)\rightarrow(\alpha\rightarrow\alpha)$
в типизированном исчислении по Чёрчу класс двуместных функций ограничен 
расширенными полиномами.
\end{theorem}

\begin{proof}
Рассмотрим некоторый лямбда-терм 
$R a^\nu b^\nu f^{\alpha\rightarrow\alpha}: \alpha\rightarrow\alpha$, 
здесь $a$ и $b$ --- переменные, содержащие чёрчевские нумералы (аргументы функции), 
$f$ --- первый аргумент чёрчевского нумерала.

Согласно свойству слабой нормализации, данный терм имеет нормальную форму $N$. 
Рассмотрим ее. Заметим, что если $T^\tau$ --- подтерм $N$, то он обязан соответствовать
одному из следующих вариантов:

\begin{tabular}{ll}
Тип & Возможный вид терма\\
\hline
$\nu$ & $a$, $b$\\
$\alpha\rightarrow\alpha$ & f, $a S_1^{\alpha\rightarrow\alpha}$, $b S_1^{\alpha\rightarrow\alpha}$,
	 $\lambda x:\alpha.S_1^{\alpha\rightarrow\alpha} (S_2^{\alpha\rightarrow\alpha} \dots S_n^{\alpha\rightarrow\alpha} (y) \dots)$\\
$\alpha$ & $f$, связанная переменная $y:\alpha$, $S_1^{\alpha\rightarrow\alpha} (T^\alpha)$
\end{tabular}

Доказать это можно разбором случаев с использованием индукции и предыдущей леммы,
для примера разберем случай $h(\tau) \ge 2$.

\begin{itemize}
\item Если $h(\tau) \ge 3$, то $T=a$ или $T=b$. 
Пусть это не так, и существуют такие $P^\pi$, что $P\ne a$, $P\ne b$ и $h(\pi) \ge 3$.
Возьмем среди таких $P$ подтерм с типом максимальной глубины.
Однако, по лемме в нем неизбежно
найдется такой $S^\sigma$, что $h(\sigma)>h(\pi)$, что противоречит максимальности $h(\pi)$.

\item Если $h(\tau) = 2$, то $\tau$ имеет вид либо $\alpha\rightarrow(\alpha\rightarrow\alpha)$,
либо $(\alpha\rightarrow\alpha)\rightarrow\alpha$. По лемме найдется такой
$S^\sigma$, что $\sigma=\tau\rightarrow\rho$ или $\sigma=\rho\rightarrow\tau$.
В любом из случаев не найдется такого $\rho$, что $\nu = \sigma$, то есть $S\ne a$ и
$S\ne b$, что невозможно по предыдущему пункту.
\end{itemize}

Теперь рассмотрим весь терм $N$. Он имеет тип $\alpha\rightarrow\alpha$. 
Введем обозначение $\overline{m}=\lambda f:\alpha\rightarrow\alpha.\lambda x:\alpha.f^m x$.
Покажем по индукции, что $N[a:=\overline{m}, b :=\overline{n}]=_\beta \lambda fx.f^{\overline{E(m,n)}} x$.

Т.о., в выражении возможны 3 только типа элементарных формул, типов $\alpha$, $\alpha\rightarrow\alpha$ и
$(\alpha\rightarrow\alpha)\rightarrow(\alpha\rightarrow\alpha)$.

Про вариант $(\alpha\rightarrow\alpha)\rightarrow(\alpha\rightarrow\alpha)$ мы уже разобрались, теперь
рассмотрим $\alpha\rightarrow\alpha$.

Это может быть либо $f$, либо $a S^{\alpha\rightarrow\alpha}$, либо $b S^{\alpha\rightarrow\alpha}$,
либо их комбинация: 
$\lambda x:\alpha.S_1^{\alpha\rightarrow\alpha}(S_2^{\alpha\rightarrow\alpha}\dots S_n^{\alpha\rightarrow\alpha}(z)))$,


\end{proof}

\subsection{Типизируемость}

