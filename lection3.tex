\section{Просто типизированное лямбда-исчисление}

Как мы уже убедились из парадокса Карри, лямбда-исчисление - это слишком мощная 
конструкция. Чтобы ее упорядочить, предпринимались многочисленные попытки.
Сперва мы рассмотрим просто типизированное лямбда-исчисление.

Идея восходит к типизации теории множеств. Там, чтобы избежать парадоксов,
можно предложить каждому множеству приписать некоторое значение (тип). Например,
мы можем приписать пустому множеству тип $0$, множеству, состоящему только из 
пустого - тип $1$, и вообще, множеству, состоящему из типов $a_1, a_2, \dots$
тип $Upb_{Ord} \{a_1, a_2, \dots\}$. При этом некоторые 

Здесь мы поступим похоже, только элементарных значений (и, соответственно, 
типов) у нас будет не одно, а счетное количество.

\begin{definition}
Тип --- это:
\begin{itemize}
\item Элементарный тип --- маленькая греческая буква ($\alpha, \beta, \dots$)
\item Составной тип. Если $\tau$ и $\sigma$ --- некоторые типы, то запись 
вида $\tau \rightarrow \sigma$ --- это также некоторый тип.
\end{itemize}
\end{definition}

Существует два основных стиля типизации лямбда-исчисления --- по Чёрчу и 
по Карри.


\subsection{Изоморфизм Карри-Ховарда}

Теперь мы готовы показать, что просто типизированное лямбда-исчисление в некотором
смысле изоморфно импликационному фрагменту интуиционистской логики.

Заметим сперва, что $T$ содержит в точности те же формулы, что и введенный в предыдущем
параграфе язык.
                 
\begin{theorem}(Об изоморфизме Карри-Ховарда)
\begin{enumerate}
\item Если $\Gamma \vdash_\textt{ч} M:\phi$, то $types(\Gamma) \vdash \phi$.
\item Если $\Gamma \vdash \phi$, то найдется такой $M \in \Lambda_T$, что
$\{x_\phi | \phi\in\Gamma\}\vdash_\textt{ч} M:\phi$.
\end{enumerate}
\end{theorem}

\begin{proof}
Доказательство обоих частей теоремы несложно, но мы приведем доказательство второй части
из методических соображений.

Покажем существование $M$ индукцией по структуре доказательства $\Gamma\vdash\phi$.
Для этого рассмотрим заключительное правило и разберем случаи.

\begin{itemize}
\end{itemize}
\end{proof}

\section{Лямбда-исчисление по Карри}

Существует второй вариант исчисления.
Главное его отличие --- в отсутствии типов при указании переменных в лямбда-термах.
Правила типизации:
 

Принципиальных отличий нет, легко показать следующую теорему:
\begin{theorem}
Пусть отображение $Er: \Lambda_T \rightarrow \Lambda$ задано так:
$Er(\lambda x:\sigma.A) = \lambda x.Er(A)$. Тогда (стирание):

\begin{enumerate}
\item Если $M\rightarrow_\beta N$, то $Er(M)\rightarrow_\beta Er(N)$
\item Если $\Gamma\vdash_\texttt{ч} M:\alpha$, то $\Gamma\vdash_\texttt{к} Er(M):\alpha$.
\end{enumerate}

Поднятие:
\begin{enumerate}
\item Если $M\rightarrow_\beta N$, то для любого $M_T \in \Lambda_T$, такого, что $Er(M_T) = M$,
найдется $N_T \in \Lambda_T$, такой, что $Er(N_T) = T$ и $M_T \rightarrow_\beta N_T$.
\item Если $\Gamma\vdash_\texttt{к} M:\alpha$, то найдется такой $M_T\in\Lambda_T$, 
что $Er(M_T)=M$ и $\Gamma\vdash_\texttt{ч} N:\alpha$.
\end{enumerate}
\end{theorem}

\begin{proof}
Упражнение.
\end{proof}

Также, легко доказать аналоги теорем Черча-Россера и теоремы о нормализации.

Однако, несмотря на сходство, есть и отличие --- типизация по Карри несколько более
широкая. А именно, если $\Gamma\vdash M:\sigma$ и $\Gamma\vdash M:\tau$, 
то из этого не следует $\sigma=\tau$. Скажем, справедливо
$\vdash_\texttt{к}\lambda x.x : \alpha\rightarrow\alpha$ и 
$\vdash_\texttt{к}\lambda x.x : \beta\rightarrow\beta$.

