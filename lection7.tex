\section{Основные задачи}

Можно задаться вопросом: что мы можем получить с этой теории?
Традиционно рассматривают следующие три задачи:

\begin{enumerate}
\item Задача проверки типов --- проверить, выполнено ли $\Gamma \vdash M:\sigma$ для
данных $\Gamma$, $M$ и $\sigma$.
\item Задача восстановления (синтеза) типов (типизируемости) --- проверить, возможно ли для
данного лямбда-выражения $M$ найти такие $\Gamma$ и $\sigma$, что $\Gamma \vdash M:\sigma$.
\item Задача населенности типа --- проверить, найдется ли для данного типа $\sigma$ контекст 
$\Gamma$ и терм $M$, такой, что $\Gamma \vdash M:\sigma$.
\end{enumerate}

Для просто типизируемого лямбда-исчисления существует алгоритмическое решение для всех
трех задач. Мы, впрочем, внимательно рассмотрим только задачу синтеза типов.
%Сейчас мы познакомимся с ним.

%\section{Синтез типа и обитаемость типа}

%Прежде чем перейти к более сложным типовым системам, нам осталось ответить на важный
%вопрос о наличии эффективных процедур, позволяющих определить, существует ли лямбда-выражение,
%имеющее некоторый тип в некотором контексте: $\Gamma\vdash ?:\sigma$

%\begin{definition}
%Мы будем называть некоторый тип $\sigma$ в контексте $\Gamma$ \emph{обитаемым}, если 
%найдется такое выражение $M$, что $\Gamma\vdash M:\sigma$.
%\end{definition}

%\begin{theorem}
%Задача определения обитаемости типа --- разрешима.
%\end{theorem}

%\begin{proof}
%Для доказательства предоставим разрешающий алгоритм.
%\end{proof}

\section{Изоморфизм Карри-Ховарда}


\subsection{Варианты просто типизированного исчисления}
%попробуем расширить понятие типа экстенсивно: через добавление новых связок, не изменяя
%порядка исчисления. Естественные кандидаты здесь --- конъюнкция и дизъюнкция, для которых 
%изоморфизм Карри-Ховарда предлагает следующие аналоги:

\begin{tabular}{lll}
Конструкция&Связка&Операции\\
\hline
Упорядоченная пара & $\alpha\&\beta$ & $\pi_1: \alpha\&\beta\rightarrow\alpha$\\
	& & $\pi_2: \alpha\&\beta\rightarrow\beta$\\
	& & $\langle\alpha,\beta\rangle: \alpha\rightarrow\beta\rightarrow\alpha\&\beta$\\
Алгебраический тип & $\alpha\vee\beta$ & $in_1: \alpha\rightarrow\alpha\vee\beta$\\
	& & $in_2: \beta\rightarrow\alpha\vee\beta$\\
	& & $case: (\alpha\rightarrow\gamma)\rightarrow(\beta\rightarrow\gamma)\rightarrow\alpha\vee\beta\rightarrow\gamma$
\end{tabular}

\section{Исчисление 1-го порядка}

Более радикальный путь усиления теории --- рассмотрение исчислений 1-го и высших порядков.
Подробно на теориях 1-го порядка мы останавливаться не будем, единственное, отметим, что
такая теория будет требовать определение выражений двух сортов: предметных и логических.
Аналогом с точки зрения изоморфизма Карри-Ховарда для логических значений будут типы,
а предметными выражениями могут быть любые выражения над не-типовыми значениями: например,
над строками, целыми числами и т.п. 

