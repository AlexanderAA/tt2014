\documentclass[12pt,a4paper,oneside]{book}
\usepackage[utf8]{inputenc}
\usepackage[english,russian]{babel}
\begin{document}

\begin{center}
\begin{Large}Программа курса <<Теория типов>>\end{Large}\\
ИТМО, группы 2537-2539, весна 2014 г.
\end{center}

\begin{enumerate}
\item Бестиповое лямбда-исчисление. Общие определения, теорема Чёрча-Россера.
\item Булевские значения, чёрчевские нумералы, упорядоченные пары, 
алгебраические типы. Нормальный и аппликативный порядок редукций.
Бета-эквивалентность и Y-комбинатор. Парадокс Карри.
\item Комбинаторы S, K, I. Выразимость лямбда-выражения в базисе SKI.
\item Просто типизированное лямбда-исчисление. Исчисление по Чёрчу и по Карри.
Основные леммы, изоморфизм Карри-Ховарда. Нетипизируемость Y-комбинатора.
\item Задачи проверки типа, обитаемости типа в просто типизированном лямбда-исчислении.
\item Задача нахождения типа просто типизированного лямбда-выражения.
\item Слабая и сильная нормализация.
\item Теорема о классе арифметических функций, представимых в просто типизированном лямбда-исчислении.
\item Интуиционистское исчисление предикатов. Модели Крипке для него.
\item Обобщенные типовые системы. Типы, рода, сорта. Лямбда-куб.
\item Система F. Изоморфизм Карри-Ховарда для системы F.
Выразимость связок через импликацию и квантор всеобщности в логике 2-го порядка.
Упорядоченные пары, алгебраические и экзистенциальные типы.
\item Неразрешимость задачи нахождения типа выражения в системе F.
\item Типовая система Хиндли-Милнера, алгоритм W.
\end{enumerate}

\end{document}
