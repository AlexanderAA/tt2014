\documentclass[12pt,a4paper,oneside]{book}
\usepackage[utf8]{inputenc}
\usepackage[english,russian]{babel}
\begin{document}

\begin{center}
\begin{Large}Программа курса <<Теория типов>>\end{Large}\\
ИТМО, группы 2536-2539, весна 2015 г.
\end{center}

\begin{enumerate}
\item Бестиповое лямбда-исчисление. Общие определения, теорема Чёрча-Россера.
\item Булевские значения, чёрчевские нумералы, упорядоченные пары, 
алгебраические типы. Нормальный и аппликативный порядок редукций.
Бета-эквивалентность и $Y$-комбинатор. Парадокс Карри.
%\item Комбинаторы S, K, I. Выразимость лямбда-выражения в базисе SKI.
\item Просто типизированное лямбда-исчисление. Исчисление по Чёрчу и по Карри.
Основные леммы, изоморфизм Карри-Ховарда. Нетипизируемость $Y$-комбинатора.
\item Задачи проверки типа, обитаемости типа в просто типизированном лямбда-исчислении.
Алгоритм унификации. Алгоритм поиска типа для лямбда-выражения.
%\item Задача нахождения типа просто типизированного лямбда-выражения.
\item Сильная нормализация.
\item Теорема о классе арифметических функций, представимых в просто типизированном лямбда-исчислении.
%\item Интуиционистское исчисление предикатов. Модели Крипке для него.
\item Интуиционистское исчисление предикатов второго порядка. Модели Крипке для него.
Выразимость всех связок через импликацию и квантор всеобщности в логике второго порядка.
\item Система $F$. Изоморфизм Карри-Ховарда для системы $F$.
Ложь, упорядоченные пары, алгебраические и экзистенциальные типы.
\item Типовая система Хиндли-Милнера, алгоритм $W$.
\item Вывод типов в системе Хиндли-Милнера с использованием ограничений.
\item Обобщенные типовые системы. Типы, рода, сорта. Лямбда-куб.
\item Линейные и уникальные типы. Комбинаторы $BCKIS$ и их типы.
\item Типовая система $F_{<:}$
%\item Неразрешимость задачи нахождения типа выражения в системе F.
\end{enumerate}

\end{document}
